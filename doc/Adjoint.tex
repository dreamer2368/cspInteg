\documentclass[11pt]{article}

\usepackage[dvips]{graphicx}

\usepackage{float}
\usepackage{psfrag}
\usepackage{amsmath,amssymb,overcite,rotating,dcolumn,texdraw,tabularx,colordvi}
\usepackage[usenames]{color}
\usepackage[nooneline,tight,raggedright]{subfigure}

\usepackage{wrapfig}
\usepackage{pstricks,enumerate}

\usepackage[font=footnotesize,format=plain,labelfont=bf]{caption}

\usepackage{hyperref}

%\captionsetup{labelfont={color=Brown,bf},textfont={color=BurntOrange}}

\definecolor{myTan}{rgb}{.7,0.4,.15}
\captionsetup{labelfont={color=brown,bf},textfont={color=myTan}}

\newcommand{\entry}[1]{\mbox{\sffamily\bfseries{#1:}}\hfil}%

\setlength{\marginparwidth}{.65in}
\def\margcomment#1{\Red{$\bullet$}\marginpar{\raggedright \Red{\tiny #1}}}

\makeatletter
\renewcommand{\section}{\@startsection
{section}%
{0}%
{0mm}%
{-0.35\baselineskip}%
{0.01\baselineskip}%
{\normalfont\Large\bfseries\color{brown}}}%
\makeatother

\makeatletter
\renewcommand{\subsection}{\@startsection
{subsection}%
{1}%
{0mm}%
{-0.35\baselineskip}%
{0.1\baselineskip}%
{\normalfont\large\bfseries\color{brown}}}%
\makeatother


\makeatletter
\renewcommand{\subsubsection}{\@startsection
{subsubsection}%
{1}%
{0mm}%
{-0.5\baselineskip}%
{0.3\baselineskip}%
{\normalfont\normalsize\itshape\centering}}%
\makeatother

%\renewcommand{\topfraction}{0.0}
\renewcommand{\textfraction}{0.0}
\renewcommand{\floatpagefraction}{0.7}


\setlength{\oddsidemargin}{0.0in}
\setlength{\textwidth}{6.5in}
\setlength{\topmargin}{-0.5in}
\setlength{\footskip}{0.30in}
\setlength{\textheight}{9.0in}
\setlength{\headheight}{0.2in}
\setlength{\headsep}{0.3in}

\def\Dpartial#1#2{ \frac{\partial #1}{\partial #2} }
\def\Dparttwo#1#2{ \frac{\partial^2 #1 }{ \partial #2^2} }
\def\Dpartpart#1#2#3{ \frac{\partial^2 #1}{ \partial #2 \partial #3} }
\def\Dnorm#1#2{ \frac{d #1 }{ d #2} }
\def\Dnormtwo#1#2{ \frac{d^2 #1}{  d #2 ^2} }
\def\Dtotal#1#2{ \frac{D #1 }{ D #2} }

\def\eps{\varepsilon}

\newcommand\eqsp[2]{
\begin{equation#1}
\begin{split}
#2
\end{split}
\end{equation#1}
}

\newcommand{\cH}{\mathcal{H}}
\newcommand{\cO}{\mathcal{O}}

\newcommand{\dcJ}{\,\delta\mathcal{J}}

\newcommand{\da}{\,\delta a}

\newcommand{\bx}{\mathbf{x}}
\newcommand{\bt}{\mathbf{\theta}}
\newcommand{\bI}{\mathbf{I}}
\newcommand{\bP}{\mathbf{P}}
\newcommand{\bF}{\mathbf{F}}
\newcommand{\bV}{\mathbf{V}}
\newcommand{\bU}{\mathbf{U}}
\newcommand{\bJ}{\mathbf{J}}

\newcommand{\inprod}[2]{\left\langle #1,#2 \right\rangle}

\newcommand{\hR}{{\hat{R}}}

\newcommand{\vF}{{\vec{F}}}

\newcommand{\dA}{{\,\delta A}}

\newcommand{\us}{u^*}

\newcommand{\bxi}{\boldsymbol{\xi}}

\newcommand{\bbK}{\hat{\mathbb{K}}}

\newcommand{\myint}{\int_0^L\int_0^{t_1}}
\newcommand{\dxdt}{\; dxdt}

\def\bdash{\hbox{\drawline{4}{.5}\spacce{2}}}
\def\spacce#1{\hskip #1pt}
\def\drawline#1#2{\raise 2.5pt\vbox{\hrule width #1pt height #2pt}}
\def\dashed{\bdash\bdash\bdash\bdash\nobreak\ }
\def\solid{\drawline{24}{.5}\nobreak\ }
\def\square{${\vcenter{\hrule height .4pt 
              \hbox{\vrule width .4pt height 3pt \kern 3pt \vrule width .4pt}
          \hrule height .4pt}}$\nobreak\ }
\def\solidsquare{${\vcenter{\hrule height 3pt width 3pt}}$\nobreak\ }


\renewcommand{\thefootnote}{\fnsymbol{footnote}}

 \renewcommand{\topfraction}{0.9}
    \renewcommand{\bottomfraction}{0.8}	
    \setcounter{topnumber}{2}
    \setcounter{bottomnumber}{2}
    \setcounter{totalnumber}{4}     % 2 may work better
    \setcounter{dbltopnumber}{2}    % for 2-column pages
    \renewcommand{\dbltopfraction}{0.9}	% fit big float above 2-col. text
    \renewcommand{\textfraction}{0.07}	% allow minimal text w. figs
    \renewcommand{\floatpagefraction}{0.7}	% require fuller float pages
    \renewcommand{\dblfloatpagefraction}{0.7}	% require fuller float pages

\setlength{\parindent}{0.25in}
\setlength{\parskip}{2.0ex}

\title{Adjoint formulation for manifold-reduced system}
\author{Seung Whan Chung}

%%%%%%%%%%%%%%%%%%%%%%%%%%%%%%%%%%%%%%%%%%%%%
\begin{document}
\maketitle

\subsection*{Original governing equation}
We refer to the original governing equation as a priori projected system:
\eqsp{}{
\Dnorm{\bx}{t} &= \bP(\bx;\bt)\cdot\bF(\bx;\bt)\\
\bP(\bx;\bt) &= \bI - \bV_f\bU_f\\
\bJ = \nabla_x\bF &= 
\begin{bmatrix}
\bV_s & \bV_f
\end{bmatrix}
\begin{bmatrix}
\Lambda_{ss} & 0 \\
0 & \Lambda_{ff} \\
\end{bmatrix}
\begin{bmatrix}
\bU_s \\
\bU_f
\end{bmatrix}
\label{e1}
}
where $\bt\in \mathbb{R}^{N_c}$ is the control parameter. For other notations, refer to the original document.

\subsection*{Observable and its sensitivity}
We have the observable as a functional of state variable $\bx$:
\eqsp{*}{
\cO[\bx] = \int_0^T o(\bx) dt,
}
and we would like to measure its sensitivity to control parameters,
\eqsp{}{
\nabla_{\bt}\cO &= \int_0^T \nabla_xo\cdot\nabla_{\bt}\bx\;dt + \int_0^T \nabla_{\bt}o\;dt\\
&= \inprod{\nabla_xo}{\nabla_{\bt}\bx} + \int_0^T \nabla_{\bt}o\;dt,
\label{e2}
}
where the inner product is defined on $\mathbb{R}^{N}$,
\eqsp{*}{
\inprod{\mathbf{a}}{\mathbf{b}} = \int_0^T a_ib_i\;dt
}
with Einstein notation for indices.

\subsection*{Sensitivity equation}
To obtain the sensitivity (\ref{e2}), the sensitivity of the solution $\nabla_{\bt}\bx$ needs to be computed.
It can be obtained by solving the linearized equation from (\ref{e1}).
From now on the equations are expressed in index notation to avoid the ambiguity.
\begin{itemize}
\item $i$, $j$, $k$: the index on $\mathbb{R}^N$ space
\item $c$: the index on control space $\mathbb{R}^{N_c}$
\item $l$: the index on fast manifold space $\mathbb{R}^{N_f}$
\end{itemize}
\eqsp{}{
\Dnorm{}{t}\partial_cx_i &= \partial_c\left( P_{ij}F_j \right)\\
&= \partial_k\left( P_{ij}F_j \right)\partial_cx_k + \partial_c\left( P_{ij}F_j \right)\\
&= \left[ \partial_kP_{ij}\cdot F_j + P_{ij}\cdot\partial_kF_j \right]\partial_cx_k + \partial_cP_{ij}F_j + P_{ij}\partial_cF_j
\label{e3}
}
\eqsp{}{
\partial_kP_{ij} &= -\partial_kV_{il}U_{lj} - V_{il}\partial_kU_{lj}\\
\partial_cP_{ij} &= -\partial_cV_{il}U_{lj} - V_{il}\partial_cU_{lj}\\
\label{e4}
}

\subsection*{Adjoint formulation}
We define a constraint functional $\cH$ as the inner product of adjoint variable and (\ref{e3}),
\eqsp{}{
\cH &= \inprod{\bx^{\dagger}}{\Dnorm{}{t}\partial_cx_i - \left[ \partial_kP_{ij}\cdot F_j + P_{ij}\cdot\partial_kF_j \right]\partial_cx_k - \partial_cP_{ij}F_j - P_{ij}\partial_cF_j }\\
&\equiv 0.
\label{e5}
}
Since $\cH$ is always equivalent to 0 as (\ref{e3}) holds, we add (\ref{e5}) to the sensitivity (\ref{e2}),
\eqsp{*}{
\nabla_{\bt}\cO &= \inprod{\nabla_xo}{\nabla_{\bt}\bx} + \int_0^T \nabla_{\bt}o\;dt = \inprod{\nabla_xo}{\nabla_{\bt}\bx} + \cH + \int_0^T \nabla_{\bt}o\;dt
}
which can be expressed in index-wise notation,
\eqsp{*}{
\nabla_{\bt}\cO &= \inprod{\nabla_xo}{\nabla_{\bt}\bx} + \cH + \int_0^T \nabla_{\bt}o\;dt\\
&= \inprod{\partial_io}{\partial_cx_i}
+ \inprod{x_i^{\dagger}}{\Dnorm{}{t}\partial_cx_i}
- \inprod{x_i^{\dagger}}{\left[ \partial_kP_{ij}\cdot F_j + P_{ij}\cdot\partial_kF_j \right]\partial_cx_k}\\
&\qquad\qquad\qquad - \inprod{x_i^{\dagger}}{\partial_cP_{ij}F_j} - \inprod{x_i^{\dagger}}{P_{ij}\partial_cF_j} + \int_0^T \partial_co\;dt
}
Our purpose of adjoint formulation is to avoid calculation of $\partial_cx_k$.
The adjoint formulation is required for first two terms of $\cH$:
\begin{itemize}
\item $\inprod{x_i^{\dagger}}{\Dnorm{}{t}\partial_cx_i}$
\eqsp{*}{
\inprod{x_i^{\dagger}}{\Dnorm{}{t}\partial_cx_i} &= \int_0^T x_i^{\dagger}\Dnorm{}{t}\partial_cx_i\;dt\\
&= x_i^{\dagger}\partial_cx_i\bigg|_{0}^{T} - \int_0^T \Dnorm{}{t}x_i^{\dagger}\partial_cx_i\;dt\\
&= x_i^{\dagger}\partial_cx_i\bigg|_{0}^{T} - \inprod{\Dnorm{}{t}x_i^{\dagger}}{\partial_cx_i}
}
\item $\inprod{x_i^{\dagger}}{\left[ \partial_kP_{ij}\cdot F_j + P_{ij}\cdot\partial_kF_j \right]\partial_cx_k}$
\eqsp{*}{
&\inprod{x_i^{\dagger}}{\left[ \partial_kP_{ij}\cdot F_j + P_{ij}\cdot\partial_kF_j \right]\partial_cx_k}\\
&= \int_0^T x_i^{\dagger}\left[ \partial_kP_{ij}\cdot F_j + P_{ij}\cdot\partial_kF_j \right]\partial_cx_k\;dt\\
&= \int_0^T x_k^{\dagger}\left[ \partial_iP_{kj}\cdot F_j + P_{kj}\cdot\partial_iF_j \right]\partial_cx_i\;dt\\
&= \inprod{ \left[ \partial_iP_{kj}\cdot F_j + P_{kj}\cdot\partial_iF_j \right]x_k^{\dagger} }{ \partial_cx_i }
}
where nothing except the index exchange between $i$ and $k$ has been done,
which corresponds to the transpose of the matrix $\in\mathbb{R}^N\times\mathbb{R}^N$.
\end{itemize}
Substituting these two terms, we obtain the dual expression for the sensitivity,
\eqsp{*}{
\nabla_{\bt}\cO &= \inprod{\nabla_xo}{\nabla_{\bt}\bx} + \cH + \int_0^T \nabla_{\bt}o\;dt\\
&= \inprod{\partial_io}{\partial_cx_i}
+ \inprod{x_i^{\dagger}}{\Dnorm{}{t}\partial_cx_i}
- \inprod{x_i^{\dagger}}{\left[ \partial_kP_{ij}\cdot F_j + P_{ij}\cdot\partial_kF_j \right]\partial_cx_k}\\
&\qquad\qquad\qquad - \inprod{x_i^{\dagger}}{\partial_cP_{ij}F_j} - \inprod{x_i^{\dagger}}{P_{ij}\partial_cF_j} + \int_0^T \partial_co\;dt\\
&= \inprod{\partial_io}{\partial_cx_i}
- \inprod{\Dnorm{}{t}x_i^{\dagger}}{\partial_cx_i}
- \inprod{ \left[ \partial_iP_{kj}\cdot F_j + P_{kj}\cdot\partial_iF_j \right]x_k^{\dagger} }{ \partial_cx_i }\\
&\qquad +x_i^{\dagger}\partial_cx_i\bigg|_{0}^{T}
- \inprod{x_i^{\dagger}}{\partial_cP_{ij}F_j}
- \inprod{x_i^{\dagger}}{P_{ij}\partial_cF_j}
+ \int_0^T \partial_co\;dt\\
&= \inprod{-\Dnorm{}{t}x_i^{\dagger} - \left[ \partial_iP_{kj}\cdot F_j + P_{kj}\cdot\partial_iF_j \right]x_k^{\dagger} + \partial_io }{\partial_cx_i}\\
&\qquad +x_i^{\dagger}\partial_cx_i\bigg|_{0}^{T}
- \inprod{x_i^{\dagger}}{\partial_cP_{ij}F_j}
- \inprod{x_i^{\dagger}}{P_{ij}\partial_cF_j}
+ \int_0^T \partial_co\;dt
}
First term in the equation above is now the new constraint functional, i.e. the adjoint equation,
and the other three terms are equivalent to the sensitivity $\nabla_\theta\cO$.
As long as the adjoint equation equals to 0, $\partial_cx_i$ needs not be computed.
Regarding the second term $x_i^{\dagger}\partial_cx_i$,
we don't really have the information about $\partial_c\bx$ at $T$,
nor use the final condition of $\bx$ as a control parameter.
This provides us with the final condition for $\bx^{\dagger}$,
\eqsp{*}{
\bx^{\dagger} = 0\qquad\qquad \text{at }t=T
}
Therefore, the sensitivity $\nabla_\theta\cO$ can be computed by solving the following adjoint system:
\eqsp{*}{
\Dnorm{}{t}x_i^{\dagger} &= - \left[ \partial_iP_{kj}\cdot F_j + P_{kj}\cdot\partial_iF_j \right]x_k^{\dagger} + \partial_io\\
\partial_kP_{ij} &= -\partial_kV_{il}U_{lj} - V_{il}\partial_kU_{lj}\\
\bx^{\dagger} &= 0\qquad\qquad \text{at }t=T
}
\eqsp{*}{
\partial_c\cO &= - x_i^{\dagger}\partial_cx_i\bigg|_{t=0}
- \inprod{x_i^{\dagger}}{\partial_cP_{ij}F_j}
- \inprod{x_i^{\dagger}}{P_{ij}\partial_cF_j}
+ \int_0^T \partial_co\;dt\\
\partial_cP_{ij} &= -\partial_cV_{il}U_{lj} - V_{il}\partial_cU_{lj}
}

\subsection*{Summary}
\begin{itemize}
\item Governing equation
\eqsp{}{
\Dnorm{\bx}{t} &= \bP(\bx;\bt)\cdot\bF(\bx;\bt)\\
\bP(\bx;\bt) &= \bI - \bV_f\bU_f\\
\bJ = \nabla_x\bF &= 
\begin{bmatrix}
\bV_s & \bV_f
\end{bmatrix}
\begin{bmatrix}
\Lambda_{ss} & 0 \\
0 & \Lambda_{ff} \\
\end{bmatrix}
\begin{bmatrix}
\bU_s \\
\bU_f
\end{bmatrix}
\label{g}
}
\item Observable and its sensitivity
\eqsp{}{
\cO[\bx;\bt] &= \int_0^T o(\bx;\bt) dt\\
\nabla_{\bt}\cO &= \inprod{\nabla_xo}{\nabla_{\bt}\bx} + \int_0^T \nabla_{\bt}o\;dt
\label{o}
}
\item Adjoint equation
\eqsp{}{
\Dnorm{}{t}x_i^{\dagger} &= - \left[ \partial_iP_{kj}\cdot F_j + P_{kj}\cdot\partial_iF_j \right]x_k^{\dagger} + \partial_io\\
\partial_kP_{ij} &= -\partial_kV_{il}U_{lj} - V_{il}\partial_kU_{lj}\\
\bx^{\dagger} &= 0\qquad\qquad \text{at }t=T
\label{adj}
}
\begin{itemize}
\item $i$, $j$, $k$: the index on $\mathbb{R}^N$ space
\item $c$: the index on control space $\mathbb{R}^{N_c}$
\item $l$: the index on fast manifold space $\mathbb{R}^{N_f}$
\end{itemize}
\item Adjoint-based sensitivity
\eqsp{}{
\partial_c\cO &= - x_i^{\dagger}\partial_cx_i\bigg|_{t=0}
- \inprod{x_i^{\dagger}}{\partial_cP_{ij}F_j}
- \inprod{x_i^{\dagger}}{P_{ij}\partial_cF_j}
+ \int_0^T \partial_co\;dt\\
\partial_cP_{ij} &= -\partial_cV_{il}U_{lj} - V_{il}\partial_cU_{lj}
\label{adj_grad}
}
\end{itemize}

\subsection*{Comments on solving the adjoint equation}
In solving the adjoint equation (\ref{adj}), we may first think of the same manifold-reduction as we do for the governing equation.
That is, we can decompose $\bx^{\dagger}$ into slow and fast manifolds according to the matrix
\eqsp{*}{
- \left[ \partial_iP_{kj}\cdot F_j + P_{kj}\cdot\partial_iF_j \right],
}
and reduce the system onto slow manifolds.\\
Although, this procedure may not be necessary and rather redundant.
On this matter you should have a better knowledge,
but I guess manifold reduction is usually required for a nonlinear system which is very stiff until it reaches the slow manifolds.
On the other hand, the adjoint system (\ref{adj}), although time-variant,
it is basically a linear system.
Of course I cannot guarantee that its stiffness is always smaller than that of the original nonlinear system.
But I hope a direct time-integration for (\ref{adj}) may be possible in an efficient way,
but simpler than just following the same decomposition for nonlinear systems.

\end{document}